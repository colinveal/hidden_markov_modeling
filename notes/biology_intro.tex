\section{DNA}
A DNA molecule consists of a long string of linked nucleotides.
Each nucleotide is composed of a nitrogenous base, five carbon sugar and a 
phosphate group. There are four different bases. The pyrimidines are
thymine (T) and cytosine (C). The purines are adenine (A) and guanine (G).
The DNA occurs as a double helix composed of two polynucleotide strands
with bases facing inwards. The two single strands run antiparallel and are connected
by hydrogen bonding between complementary bases. Because the two strands are complementary,
we can represent a DNA molecule by a sequence of bases in a single strand.
The top of the strand is known as the $5'$ and the bottom as $3'$. Sequences are always read from
$5'$ to $3'$.


A gene is a specific sequence of codons that carry the information required to construct proteins. Chromosomes are large segments of DNA which contain many genes. The size, structure and the numbers
of chromosomes vary among different organisms. Taken together the organism's   chromosome make up the genome.

\section{Copy Number Variation CNV}

Copy number variants (CNVs) are alternations of DNA of a genome that results
in the cell having a less or more than two copies of segments of the DNA \citep{cai2012}.
CNVs correspond to relatively large regions of the genome, ranging from about one kilobase
to several megabses that are deleted or duplicated \cite{cai2012}.

CNVs can be discovered by cytogenetic techniques, array comparative genomic
hybridization and by single nucleotide polymorphism (SNP) arrays see \citep{cai2012} and
references therein. Furtheremore, CNVs can be identified by next-generation sequencing (NGS) 
in high resolution \cite{cai2012}. NGS
can generate millions of short sequence reads along the whole human genome. When these
short reads are mapped to the reference genome, both distances of paired-end data 
and read-depth (RD) data can reveal the possible structure variations of the target genome \cite{cai2012}
Another important source of information useful for inferring CNVs from reads alignment is
the read depth (RD). The RD data are generated to count the number of reads that cover a
genomic location or a small bin along the genome which provide important information
about the CNVs that a given individual carries \cite{cai2012}. 
When the genomic location or bin is within a deletion, one expects to observe a
smaller number of read counts or lower mapping density than the background read depth. In
contrast, when the genomic location or bin is within an insertion or duplication, one expects
to observe a larger number of read counts or higher mapping density. Therefore, these RDs
can be used to detect and identify the CNVs. The read-depth data provide more reliable
information for large CNVs and CNVs flanked by repeats, where accurate mapping reads is
difficult. The read depth data also provide information on CNVs based on the targeted
sequences where only targeted regions of the genome are sequenced \cite{cai2012}.


\section{TUF}
\label{tuf}

TUF cores in WGA sequencing samples can be identified as gaps in RD coverage. Thus, they
mimic deletions. This means that TUF cores can be represented as deletions in RD data. Thus, we can adapt
existing CNV detection models for TUF detection.


























